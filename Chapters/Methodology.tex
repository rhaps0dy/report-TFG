%DIJOUS
\chapter{Methodology}
\section{Reverse-engineering \acl{MR}}
We used the \acl{ALE} (\cite{bellemare2013arcade}) to take several simultaneous
screen and \ac{RAM} snapshots. We usually took 5 to 10 snapshots in the space of
1 or 2 seconds, while performing certain action in the game. Then, we looked at
the bytes that changed value from snapshot to snapshot.

We also used the debugger built-in to the Atari 2600 emulator, Stella
(\cite{stella}). Using the command \verb-breakif <condition>-, that pauses the
game and shows the debugger if a condition is met, enabled us to play and check
whether a memory position behaved as we suspected.

In table \ref{table:atari-ram} and in the list below we reproduce the layout of
the Atari 2600's main memory and what each position affects in \acl{MR}.

Some entries can be modified in the Stella debugger when the game is running,
and affect the game. They will be marked with an asterisk (\textbf{*}). The
values that are not marked editable may be editable in other circumstances, and
are probably editable in the middle of the computations within a frame. However,
their value has been observed only going back to what it was if modified between
frames.

\fxwarning{Explain the Atari's RAM, explain Panama Joe}

\begin{table}[h]
\begin{center}
\newcommand{\ram}[2]{\hyperref[ram:#1]{#2}}
\newcommand{\rame}[2]{\hyperref[ram:#1]{\textbf{#2*}}}
\begin{tabular}{c|cccccccccccccccc}
  & 0 & 1 & 2 & 3 & 4 & 5 & 6 & 7 & 8 & 9 & A & B & C & D & E & F \\
\hline
8 &   &   &   &\rame{screen}{83}&   &   &   &   &   &   &   &   &   &   &   &   \\
9 &   &   &   &\rame{score}{93}& \rame{score}{94}& \rame{score}{95}&   &   &   &   &   &   &   &   &   &   \\
A &   &   &   &   &   &   &   &   &   &   &\rame{x}{AA}&\rame{y}{AB}&   &   &   &   \\
B &   &   &   &   &   &   &   &   &   &   &   \rame{lives}{BA} & & & & & \\
C &   &   &   &\rame{inventory}{C1}&   &   &   &   &   &   &   &   &   &   &   &   \\
D &   &   &   &   &   &   &   &   &   &\rame{jump}{D6}&   &\rame{fall}{D8}&   &   &   &   \\
E &   &   &   &   &   &   &   &   &   &   &   &   &   &   &   &   \\
F &   &   &   &   &   &   &   &   &   &   &   &   &   &   &   &   \\
\end{tabular}
\end{center}
\label{table:atari-ram}
\caption{The \acs{RAM} layout for \acl{MR}.}
\end{table}

{
\newcommand{\entry}[2]{\item\label{ram:#1}\textbf{0x#2}: Not editable.}
\newcommand{\entrye}[2]{\item\label{ram:#1}\textbf{0x#2}: Editable.}
\newcommand{\n}[1]{0x#1}

\begin{enumerate}
\entrye{lives}{BA} The number of lives the player has left, that is, the number of
times the player can die and continue the game afterwards. Controls the number
of hats displayed at the top. Panama Joe starts with 5 lives. The counter can go
up to 6 without graphical problems.
\entrye{jump}{D6} The current frame of the jump. Set to \n{FF} when in the ground. Set
to \n{13} when the jump starts. If set to higher than \n{13}, the game behaves
oddly. It can be reset to whatever value at any time, causing Panama Joe to
start a jump, even in mid-air.
\entrye{fall}{D8} The current frame of the fall. Normally set to \n{00}. When falling
off an elevated ground, or off a jump, this value will begin to count up. If it
is \n{08} or higher when Panama Joe touches the ground, he will die.
\item\label{ram:score}\textbf{0x93}, \textbf{0x94}, \textbf{0x95}: Editable. The
score, represented in \acs{BCD}. This is, every nibble represents a decimal digit.

\entrye{screen}{83} The current screen. If edited, the new screen will only be
partially drawn. Exit the screen and reenter it by playing and the issue will go
away.
\entrye{x}{AA} The X position of the character. If set to the middle of the air,
Panama Joe will fall.
\entrye{y}{AB} The Y position of the character. If set to the middle of the air,
and there is a platform below, the character will not fall! Instead, it will
behave as if it was on a ladder. The Y values of the three floors that every
level has are \n{94}, \n{C0} and \n{EB}.
\entrye{inventory}{C1} The contents of the player's inventory. Each possible
object in it is associated to a bit, that is set if the object is in the
inventory. At most 6 objects can be carried without causing graphical
corruptions. The objects and their associations are:
\begin{center}
\begin{tabular}{c|c|c|c|c|c|c}
\n{80} & \n{40} & \n{20} & \n{10} & \n{08} & \n{04} & \n{02} & \n{01} \\
\hline
torch sword sword key key key key mallet \\
\end{tabular}
\end{center}
\end{enumerate}
}


\fxwarning{Talk about the RAM of the Atari having 128 bytes and the ones relevant are 0x80 0xff}


\subsection{Reward shaping\label{subsection:reward-shaping}}
\begin{equation}
  t_1 = \frac{v_{x_2}(y_1-y_2) - v_{y_2}(x_1-x_2)}{v_{y_2}v_{x_1} - v_{x_2}v_{y_1}}
\end{equation}
\section{Planning}
\subsection{\acl{IW}(1)}
\subsection{\acl{IW}(3) on location}
\subsection{\acl{IW}(3) on location with heuristic}
\section{Learning}
\subsection{Shaped tabular Sarsa}
\subsection{Task by task \acs{DQN}}
\subsection{Tabular Sarsa combined with planning}

%%% Local Variables: 
%%% mode: latex
%%% TeX-master: "../report"
%%% End: 