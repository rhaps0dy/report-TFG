\documentclass[12pt,a4paper,oneside]{report}
\usepackage{setspace}
% 1.5 line spacing
\setstretch{1.5}
\usepackage{amssymb,amsmath}
\usepackage{graphicx}
\usepackage{eufrak}
\usepackage{calc}
\usepackage[utf8]{inputenc}
\usepackage[english]{babel}
% use upquote if available, for straight quotes in verbatim environments
\IfFileExists{upquote.sty}{\usepackage{upquote}}{}
% use microtype if available
\IfFileExists{microtype.sty}{%
\usepackage{microtype}
\UseMicrotypeSet[protrusion]{basicmath} % disable protrusion for tt fonts
}{}
\usepackage[tmargin=3cm, lmargin=2.5cm, rmargin=2.5cm, bmargin=3cm]{geometry}
\usepackage{hyperref}
\usepackage{import}
\PassOptionsToPackage{usenames,dvipsnames}{color} % color is loaded by hyperref
\hypersetup{unicode=true,
            colorlinks=true,
            linkcolor=black,
            citecolor=black,
            urlcolor=blue,
            breaklinks=true}
\urlstyle{same}  % don't use monospace font for urls
\IfFileExists{parskip.sty}{%
\usepackage{parskip}
}{% else
\setlength{\parindent}{0pt}
\setlength{\parskip}{6pt plus 2pt minus 1pt}
}
\setlength{\emergencystretch}{3em}  % prevent overfull lines
\providecommand{\tightlist}{%
  \setlength{\itemsep}{0pt}\setlength{\parskip}{0pt}}
\setcounter{secnumdepth}{0}
% Redefines (sub)paragraphs to behave more like sections
\ifx\paragraph\undefined\else
\let\oldparagraph\paragraph
\renewcommand{\paragraph}[1]{\oldparagraph{#1}\mbox{}}
\fi
\ifx\subparagraph\undefined\else
\let\oldsubparagraph\subparagraph
\renewcommand{\subparagraph}[1]{\oldsubparagraph{#1}\mbox{}}
\fi

%% BIBTEX
\usepackage{csquotes}
\usepackage[nottoc,numbib]{tocbibind}
\usepackage[style=authoryear,bibencoding=ascii,backend=bibtex]{biblatex}
\addbibresource{report.bib}

%% CUSTOM COMMANDS
\newcommand{\centeredtitle}[1]{\begin{center} \large{#1}
    \addcontentsline{toc}{chapter}{#1} \end{center}}

%% DOCUMENT
\author{Adrià Garriga}
\title{Solving Montezuma's Revenge with planning and reinforcement learning}

\begin{document}
\pagenumbering{roman}
\maketitle

\newpage
\vfill
\centeredtitle{Acknowledgements}
My supervisor Anders Jonsson, for the patience and understanding showed

My good friend Daniel Furelos, for the patience and help

My family who allowed me to turn my room into a computing center
\vfill

\newpage
\vfill
\centeredtitle{Abstract}

Traditionally, methods for solving sequential decision processes (SDPs) have not
worked well with those that feature sparse feedback. Both planning and
reinforcement learning, methods for solving SDPs, have trouble with sparse
feedback.

With the rise to prominence of the Arcade Learning Environment (ALE) in the
broader research community of sequential decision processes, one SDP featuring
sparse feedback has become familiar: the Atari game Montezuma's Revenge. In this
particular game, the great amount of knowledge the human player already
possesses, and uses to find rewards, cannot be bridged by blindly exploring in a
realistic time.

We apply planning, hierarchical reinforcement learning and hybrid approaches,
combined with domain knowledge, to enable an agent to obtain better scores in
this game.

We hope that these domain-specific algorithms can inspire better approaches to
solve SDPs with sparse feedback in general.
\vfill


\newpage
\tableofcontents
\newpage
\listoffigures
\newpage
\listoftables
\newpage
\pagenumbering{arabic}

\chapter{Introduction}
  \section{Context}
  Very short history of sequential decision processes AI. Talk about Deepmind DQN Atari ,and the rise to ``fame'' of montezuma's revenge
  \section{Problem}
   the problem of sparse feedback in RL and planning, and that we tackle this one game  to try and learn something from it
  

\chapter{Related Work}
  the very related Hierarchical Deep Reinforcement Learning: Integrating Temporal Abstraction and Intrinsic Motivation

  DeepMind's Human-level control in Atari games, 

  Maybe Dani's TFG

\chapter{Theoretical Background}
The aim of this thesis is to produce a computer program that plays Montezuma's
Revenge well. This problem statement suffices for most communication purposes,
but does not give us enough understanding to reason about the problem and find
ways to solve it. We first need to develop a formal definition of all the
notions: ``to play'', ``Montezuma's Revenge'' and ``well''. Fortunately, most of
the required work has already been done, by other authors.

In this chapter we will define a mathematical model for the problem we are
facing. We will also formally define the algorithms we will use to tackle it,
without concerning ourselves with the details of their implementation on our
computing environment.

\section{Sequential Decision Process}
Let us describe the Sequential Decision Process (SDP) model. There is an
\emph{agent} that, every \emph{time step}, takes an \emph{action} in the
\emph{environment}. The environment is a process that has a \emph{state}. When
the agent takes an action, the environment's state changes. The agent also
receives a \emph{reward} when it takes an action.

More formally. There is a series of discrete time steps, $t=0,1,2,3,\dots$, in
which the agent and the environment interact. At time step $t$, environment is
in state $s_t \in \mathcal{S}$. $\mathcal{S}$ is the finite, but usually very
large, set of possible states. The agent takes an action $a_t$ from the set of
possible actions in the state, $a_t \in \mathcal{A}(s_t)$. In the next time
step, the agent receives a numerical reward $r_{t+1} \in \mathfrak{R}$, and the
environment transitions to a new state $s_{t+1} \in \mathcal{S}$.

The environment defines the set of possible states $\mathcal{S}$, the possible
actions in each state $\mathcal{A}$, the reward for each state $\mathcal{S}$ and
the rules for transitioning to the new state in each time step. The agent simply
chooses the action $a_t$ in each time step. The interaction between agent and
environment is illustrated in figure \ref{fig:agent-environment}.

\begin{figure}[hbtp]
\begin{center}
  \def\svgscale{1}
  \subimport{img/}{agent-environment.pdf_tex}
\end{center}
\caption{Diagram of interaction between agent and environment
  \cite{sutton1998introduction}\label{fig:agent-environment}}

\end{figure}

\


\section{Reinforcement Learning}
\section{Sarsa}
\section{Hierarchical Reinforcement Learning}
Theory about MAXQ and its application to the task we decomposed
\section{Learning while evaluating a different policy}
\section{Iterated Width}

\chapter{Experiments}
\section{Reverse-engineering Montezuma's Revenge}

\chapter{Approaches}
Include failed approaches or only successful ones?



\chapter{Conclusions and Future Work}
  \section{Conclusions}
  That a little domain knowledge can go very far
  \section{Future work}
  How we can generalize the results obtained here to other sparse-feedback MDPs.
  Discuss the possible meta-learner

\newpage
\printbibliography

\end{document}

Provar d'anar cap a estats no visitats
